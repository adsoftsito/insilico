\title{ Framework basado en Inteligencia Artificial para descubrir nuevos Medicamentos }
\author{
        Adolfo Centeno Tellez \\
                Departmento de Ciencias Computacionales\\
        Instituto Tecnologico y de Estudios Superiores de Monterrey\\
        Campus Puebla, \underline{Mexico}
            \and
        Gonzalo Aranda Abreu\\
        Centro de Investigacion en Ciencias Cerebrales\\
        Universidad Veracruzana\\
        Campus Xalapa, \underline{Mexico}
}
\date{\today}

\documentclass[12pt]{article}

\begin{document}
\maketitle

\begin{abstract}
Este trabajo propone la creacion de un marco de trabajo basado en Inteligencia Articial para el descubrimiento de nuevos medicamentos, usando el metodo insilico.
\end{abstract}

\section{Introduction}
En la actualidad el uso de las grandes clusters de supercomputadoras, la inteligencia artificial particularmente el Deep Learning han ayudado en la generacion de conocimiento nuevo en diversas areas de la ciencia. El presente trabajo pretende usar redes neuronales artificiales del tipo Variational Autoencoders para el descubrimiento de nuevos medicamentos usando la base de datos MOSES como principal fuente de informacion para el entrenamiento.


\section{Objetivo general}\label{Objetivo general}
Construir un marco de trabajo para simular por computadora el diseno de farmacos nuevos, que incluya servidores en la nube, software, procesos, procedimientos

\section{Objetivos especificos}\label{Objetivos especificos}
\begin{enumerate}
 \item Instalar configurar los servicios en servidores
 \item Desarrollar y/o instalar software de IA para la simulacion insilico
 \item Documentar los procesos para realizar las simulaciones
 \item Validar el experimento contra otros modelos realizados in vitro, en animales, entre otros.
 \item Publicar resultados
\end{enumerate}

\section{Cronograma de actividades}\label{Cronograma de actividades}
\begin{enumerate}

 \item Investigar el estado del arte del modelo insilico y herramientas existentes (www.insilico.com)
 \item Establecer modelo de computo en la nube para la instalacion y configuracion de las herramientas de software
 \item Entrenanamiento en herramientas tensorflow y keras basadas en python
 \item Entrenamiento en redes neuronales VAN y modelos generativos (ejemplos con rostros, flores, aves, entre otros)
 \item Instalacion del modelo de base de datos de moleculas MOSES (5,000,000 de medicamentos)
 \item Instalar, configurar modelos existentes como el modelo GENTRL de la empresa china insilico.com.
 \item Probar el modelo pharmaio
 \item Probar el modelo pharmaio xxx
 \item Seleccionar una caso de estudio para simular la generacion de un farmaco
 \item Codificar las redes neuronales para conducir el experimento
 \item Validar el experimento
 \item Documentar procesos y resultados
 \item Publicar hallazgos de la investigacion

\end{enumerate}

\bibliographystyle{abbrv}
\bibliography{main}

\end{document}
